\documentclass[a4paper, 11pt]{article}

% Math symbols, notation, etc.
% Apparently, must be loaded earlier than mathspec?
\usepackage{amsmath}
\usepackage{amssymb}
\usepackage{stmaryrd}

% Locale/encoding with XeTeX: UTF-8 by default, use fontspec
\usepackage{unicode-math}
\usepackage{polyglossia} % Modern replacement for Babel
\setmainlanguage{english}
\usepackage{csquotes} % guillemets

% TeX Gyre Pagella = URW Palladio (free Palatino) extended
%\setmainfont[Mapping=tex-text]{TeX Gyre Pagella}
\setmathfont{Asana Math}

% Other
%\usepackage{fullpage}
%\usepackage{enumerate}
%\usepackage{graphicx}

\newcommand\co[1]{\texttt{#1}}
\begin{document}

\title{Report on the netlist simulator}
\author{Nguyễn Lê Thành Dũng \and Thomas Bourgeat}

\maketitle

\section{Conversion of the provided OCaml code to Haskell code }
The first step was to convert the code provided, written in OCaml, in
Haskell. While the references and mutable constructions are idiomatics in
OCaml, we wanted to avoid it in Haskell because it is pretty much
complicated and not idiomatic. Purer the code is, better it
is. In this first part we show some example.

\subsection{Purely functional data structures}
\begin{itemize}
\item The \co{Array} are replaced by \co{IntMap} or \co{[Int]}
\item The mutables are deleted.
\end{itemize}

\subsection{Use of monads}
Sometimes, the algorithms are really imperative, and to control de flow of
our program and imitate this style of programmation, we use monads. For
example:

 we could not easily avoid the references for the topological sort, so we
used the monad \co{State} combined with \co{Maybe}, i.e The monad \co{StateT
(Map (Node a) Mark) Maybe}.


But we tried to have the maximum of pure code. In the other cases, we
mainly use monads for plumbing. 

\section{Global architecture}

\subsection{RAM}
We used a sparse representation of the RAM : \co{Data.IntMap}. And we
chose to suppose that initially, the RAM is full of $0$. 
\subsection{ROM}
We used a immutable \co{Array} structure of Haskell. Which is initially filled with a
ROM input file containing a sequences of binary digits. 

\section{Input parsing and plumbing}

\section{Discussion about performances}
We risk a lack of performance in this version 0.1, mainly due to the
abundant use of lazyness (but that's so cool to use it). But we 
justify it with:  
\begin{itemize} 
\item Our main goal was the correction of our simulator, which is clearer without
optimizations and dirty-in-haskell datastructures.
\item. We keep in mind that
in fine, we have to write a processor and simulate this processor : it
can be usefull to save all the datas for future debugging. For example, this version
can very easily be updated to print intermediate configurations, which
is helped by purity of the majority of the code.   
\item A think about a version 0.2 based on :
\begin{itemize}
\item A written-in-haskell compiler, from Netlist to C.
\item GCC -O3
\item A dynamic linking of functions compiled in C, called in Haskell,
that permit a really fast simulation. It allows us to keep the full
benefits of 
Haskell, with the full speed of C.

\end{itemize}
\end{itemize}


\end{document}
