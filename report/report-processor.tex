\documentclass[a4paper, 11pt]{article}

% Locale/encoding with XeTeX: UTF-8 by default, use fontspec
\usepackage{unicode-math}
\usepackage{polyglossia} % Modern replacement for Babel
\setmainlanguage{english}
\usepackage{csquotes} % guillemets
\usepackage{listings}
% Other
\usepackage{fullpage}
%\usepackage{enumerate}
%\usepackage{graphicx}

\begin{document}

\title{Report on the processor design}
\author{Nguyễn Lê Thành Dũng \and Thomas Bourgeat}

\maketitle

\section{A non-linear Lisp-like assembly language: S-code}

We wanted our machine language to be suited for the compilation of functional programming languages; that is, its instruction set would provide first class support for functions calls and closures. To design such a language, we looked up the research done at the MIT AI Lab in the 70's on Lisp Machines. Among the different architectures they had developed, we were directly inspired by a family of particularly unusal ones, described in \cite{SIMPLE} and \cite{SCHEME-79}.

In these architectures, the machine code is not a conventional ordered array of instructions but instead a structured tree, called S-code for its close resemblance to the S-expressions which constitute Lisp programs. Indeed, the main idea is that since Lisp, like assembly, is \emph{homoiconic} (that is, code in this language is made of the primitive data structures of the language), it could serve as a kind of \enquote{high-level assembly} for a stored-program computer.

More precisely, in our architecture, the raw material from which both code and data is built is:
\begin{itemize}
\item \emph{machine words}, which consist of
  \begin{itemize}
  \item A field which encodes a tag used to describe the type of the data. When interpreted as code, this is analogous to an opcode in classical processors.
  \item A data field which contains either immediate data or a pointer.
  \end{itemize}
\item \emph{cons cells}, which are pairs of words
\end{itemize}
Linking cons cells together with pointers gives rise to syntax trees.

For example, the following is a representation as an OCaml data structure of a S-code program which computes 21 + 21 = the answer to the universe:
\lstset{basicstyle=\small\ttfamily,language=Caml}
\begin{lstlisting}
["main",
  First(Num(21),Next(Num(21),CallFun(Global("add")))) ;
 "add",
 Sequence(First(Local(0,0),CallPrim(Zero)),
          Cond(Local(0,1),
               First(First(Local(0,1),CallPrim(Incr)),
                     Next(First(Local(0,0),CallPrim(Decr)),
                          CallFun(Global("add"))))))
]
\end{lstlisting}
(this example actually works on our interpreter).

The grammar of the S-code is not finalized yet.


\section{Global architecture of the microprocessor}

Basically, the goal of the architecture is to implement in hardware an S-code interpreter. To do so, we first wrote an interpeter in OCaml in an imperative, state-machine style using global variables. Our next goal will be to mechanically convert this interpreter to a hardware state machine using a state counter and a combinational transition function, and global registers. (Actually, we intend to \enquote{cheat} by using lookup tables to compute the transitions and the control signals. In addition to making the implementation easier, it should lead to better performance.)

The interpreter issues commands to a memory system which handles the allocation of new words and cons cells, and the dereferencing of pointers. Morally, the kind of memory needed for the interpreter is an unordered heap on which we can allocate new words without worrying. Since the primitive we can use in our netlist is actually a fixed-size RAM array, we need to create a hardware-level abstraction layer. For our purposes, a RAM chip with a counter pointing towards the next free cell is enough. Even though we won't implement a mark-and-sweep algorithm in hardware (as is done in \cite{SCHEME-79}), we do intend to make sure our clock will be theoretically able to run indefinitely.


\section{The issue of time synchronization}

A big problem not particularly tied to our choice of architecture is the necessity to handle multiple clocks : the synchronous clock of the circuit, and the real time on which we need to align the execution of the digital watch program.

In our assembly language, this function is fullfilled by a \enquote{sync} instruction which tells the simulator to pause, and resume executing at the next second. Basically, after computing the time it will be at the next second, the program will wait for it before printing the time and then looping.


\section{Netlist generation with a Haskell eDSL: Caillou }

We developed a little language named Caillou (a reference to Lava, Magma was already taken by a computer algebra system). It is a domain specific language for writing circuits, embedded in Haskell, which compiles to the netlist language specified in the project. Using this language gives us the power of a full-fledged programming language with abstraction facilities and good tooling (whereas MiniJazz doesn't even have an Emacs mode!).


\bibliographystyle{plain}
\bibliography{biblio}

\end{document}
